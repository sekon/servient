\documentclass{book}
 \title{Servient: Configuration and Usage Guide}
 \author{Harish Badrinath}
 \begin{document}
  \maketitle
  \tableofcontents{}
  \chapter{Introduction}
    \section{Prologue}
      Hello, firstly thank you for your interest in servient. \\
      \textbf{Note:}\\
      \begin{itemize}
        \item This Guide is intended to be mainly for people who deploy and use servient.\\
        \item Servient Aims to primarily service the following use case: Batching the validation of programs, which are prospective solutions to a problem statement. To put it in very simple words, you can automate the correction of say, student submitted solution to a lab program.
      \end{itemize}
    \section{Some Meta Information}
        Serivient is black box testing framework that is small, simple, flexible and light weight. Focussed on testing in a class room like environment.
        It aims to be written in standard compliant shell, so that it can be run on various shells in a consistent manner.
        It is written as part of the FOSSEE$^{\textrm{\cite{fossee}}}$ project. It was deployed in aiding the efforts of programs like, the thousand teacher program.
        The source coude of this project is on github$^{\textrm{\cite{servient-repo}}}$.

  \chapter{Changelog}
    \begin{tabular}{ | c | c | c | c | }
      \hline
      Date & Version & Change Message & Author\\ \hline
      Tue, 07 Feb 2012 23:00:31 +0530 & 0.1 & Initial Commit & Harish Badrinath\\ \hline
    \end{tabular}
  \begin{thebibliography}{99}
  \bibitem{fossee}
    FOSSEE Project, Based at Indian Institute of Bombay.\\
    Please see http://fossee.in/
   \bibitem{servient-repo}
    https://github.com/sekon/servient
  \end{thebibliography}
\end{document}
